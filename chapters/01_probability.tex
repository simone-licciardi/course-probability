% 01_probability.tex
%! TeX root = ../probability.tex

\chapter{Spazi Probabilizzati}

Il nostro obiettivo è definire una funzione (quindi, un dominio e una legge) detta "probabilità" che associ a insiemi di configurazioni del sistema un indice della loro \textit{likelyhood}\footnote{In italiano, l'espresione più vicina sarebbe verosimiglianza.}.

\subsection*{Eventi}

Formalmente, consideriamo l'insieme delle possibili configurazioni del sistema, lo chiamiamo \textit{spazio campionario}, e lo indichiamo con $\Omega$. Un suo sottoinsieme $A\subset\Omega$, e cioè un insieme di configurazioni, si dice \textit{evento}. Quindi, $\mathcal{P}(\Omega)$ è l'insieme degli eventi, e in particolare $A=\Omega$ è detto evento \textit{certo} e $A=\emptyset$ è detto evento \textit{impossibile}. 

\section{Collezioni di eventi}

Quindi la probabilità è una funzione su una collezione di eventi in $\mathbb{R}$. Poichè ci interessano solo i sottoinsiemi "sensati" dello spazio campionario, in primo luogo studiamo come strutturare il dominio della funzione.

Per costruire un insieme ci sono pochi metodi. I principali sono:
\begin{enumerate}
	\item Fare una lista degli elementi,
	\item Dato un insieme, definirne un sottoinsieme che rispetti una legge,
	\item Dato un insieme, il suo insieme delle parti,
	\item Dati due insiemi, il loro prodotto cartesiano,
	\item Dati due insiemi, la loro unione, la loro intersezione, e la loro differenza.
\end{enumerate}
Di questi è chiaro che l'unico adeguato alla costruzione di una probabilità è l'ultimo. Quindi, definiamo due collezioni di insiemi che siano \textit{stabile} (o \textit{chiusa}) rispetto a queste operazioni, rispettivamente nel contesto finito e in quello numerabile.


\subsection{Algebra e $\sigma$-algebra}

\begin{my_definition}
	La collezione $\mathcal{F} \subset \mathcal{P}(\Omega)$ si dice algebra di $\Omega$ se e solo se
	\begin{itemize}
	   	\item[(Normalization)] $\emptyset, \Omega \in \mathcal{F}$
	    	\item[(Complementation)] $A \in \mathcal{F} \implies A^\complement \in \mathcal{F}$
	    	\item[(Union \& Intersection)] $\{A_1, \dots, A_n\} \subset \mathcal{F} \implies \bigcup_{i=1}^{n} A_i \in \mathcal{F}$
	\end{itemize}
\end{my_definition}
    
Questo insieme è stabile rispetto all'unione finita, all'intersezione finita e alla complementazione.
    
\begin{my_definition}
	La collezione $\mathcal{F} \subset \mathcal{P}(\Omega)$ si dice $\sigma$-algebra di $\Omega$ se e solo se
	\begin{itemize}
	    	\item[(Normalization)] $\emptyset, \Omega \in \mathcal{F}$
	    	\item[(Complementation)] $A \in \mathcal{F} \implies A^\complement \in \mathcal{F}$
	    	\item[(Union \& Intersection)] $\{A_1, \dots\} \subset \mathcal{F} \implies \bigcup_{i=1}^{\infty} A_i \in \mathcal{F}$
	\end{itemize}
\end{my_definition}

Questo insieme è invece stabile rispetto all'unione numerabile, all'interseziona numerabile e alla complementazione.

Partiamo osservando che queste definizioni sono ridondanti grazie alla complementazione. Infatti, la prima condizione è equivalente a $\emptyset\in\mathcal{F}$ oppure a $\emptyset\in\mathcal{F}$, e la terza è equivalente alla sola stabilità rispetto all'unione o rispetto all'intersezione. Questa osservazione è operativamente comoda quando è necessario verificare che una collezione sia una $\sigma$-algebra.

\subsection{Altre classi stabili}

\section{Misure di probabilità}

% misura di probabilità finitamente additiva
% 	- (convenzione di normalizzazione) evento certo
% 	- (complementazione)
% 	- (additività finita)

% remark su perchè "finitamente"
% 	- per additività finita 
% ? remark su inadeguatezza perchè esclude algebre non finite
% 	- osservare che spazio campionario non è necessariamente finito

% definizione di sigma-algebra
% 	- casi elementari
% 	- chiusura rispetto alla complementazione
% 	- chiusura rispetto alla unione numerabile
% remark sul fatto che una sigma algebra è un'algebra, e quindi le proprietà della seconda valogono per la prima ((strictly) larger). non è necessario esempio del perchè strictly è rilevante

% ? remark su inadeguatezza per gestione sigma algebre non finite

% misura di probabilità
% 	- (convenzione di normalizzazione) evento certo
% 	- (complementazione)
% 	- (sigma-additività)==(additività numerabile)	

% caratterizzazione della terza proprietà attraverso [...?]

% proprietà elementari
% 	-
% 	-
% 	-
% 	-

% remark su forza dell'identità insiemistica vs probabilistica (forte vs debole)
% remark su normalizzazione del campionamento
% remakr principio incl escl

% esempio metrica su algebra che gode di additività, ma non sigma- additività.

% teorema: chiave

% Appendici:
% - insieme delle parti